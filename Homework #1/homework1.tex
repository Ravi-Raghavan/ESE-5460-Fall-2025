\documentclass[11pt, reqno, letterpaper, twoside]{amsart}
\linespread{1.2}
\usepackage[margin=1.25in]{geometry}

\usepackage{amssymb, bm, mathtools,physics}
\usepackage[usenames,dvipsnames,svgnames,table]{xcolor}
\usepackage[pdftex, xetex]{graphicx}
\usepackage{enumerate, setspace}
\usepackage{float, colortbl, tabularx, longtable, multirow, subcaption, environ, wrapfig, textcomp, booktabs}
\usepackage{pgf, tikz, framed}
\usepackage[normalem]{ulem}
\usetikzlibrary{arrows,positioning,automata,shadows,fit,shapes}
\usepackage[english]{babel}

\usepackage[final]{microtype}

\theoremstyle{plain}
\newtheorem{theorem}{Theorem}

\theoremstyle{definition}
\newtheorem{solution}[theorem]{Solution}
\renewcommand{\solution}{\textbf{Solution}\newline}


\usepackage{times}
\title{ESE 546\\[0.1in]
Homework 1}
\author{
Ravi Raghavan [rr1133@seas.upenn.edu],\\
Collaborators: N/A
}

\begin{document}
\maketitle

% Important LInks
% 1. https://scikit-learn.org/stable/modules/svm.html#implementation-details
% 2. https://scikit-learn.org/stable/auto_examples/svm/plot_svm_kernels.html#sphx-glr-auto-examples-svm-plot-svm-kernels-py
% 3. https://www.csie.ntu.edu.tw/~cjlin/papers/libsvm.pdf
% 4. https://scikit-learn.org/stable/auto_examples/svm/plot_rbf_parameters.html

% https://scikit-learn.org/stable/modules/svm.html#svm-mathematical-formulation
\begin{solution}[Time spent: 1 hour]
\begin{enumerate}
    \item[(a)] The slack-variable based optimization problem can be formulated as such
    \begin{align}
        \label{eq:1a-1}
            \begin{array}{ll}
\text{minimize}_{\theta, \theta_0, \xi_1, \cdots, \xi_n} & \frac{1}{2} \|\theta\|^2 +  C \sum_{i = 1}^n \xi_i \\
\text{subject to} & y_i (\theta^T x_i + \theta_0) \geq 1 - \xi_i,\, \forall i = 1, \cdots, n \\
& \xi_i \geq 0,\, \forall i = 1, \cdots, n
\end{array}
        \end{align}

    where $C$ is a hyperparameter and our optimization variables are $\theta, \theta_0, \xi_1, \cdots, \xi_n$. \\
    \noindent From \eqref{eq:1a-1}, we observe that a larger value of $C$ places a stronger penalty on slack, reducing the margin and prioritizing correct classification of training points. Conversely, a smaller value of $C$ allows more slack, resulting in a wider margin and greater tolerance for misclassifications.  

\textbf{\underline{Answer:}} The objective function is $\frac{1}{2} \|\theta\|^2 +  C \sum_{i = 1}^n \xi_i$, where $C$ is a hyperparameter as described above

    \item[(b)] In an SVM, the support samples are the \textbf{\underline{critical}} data points that determine the position/orientation of the separating hyperplane. Removing any non-support sample would NOT affect the decision boundary. In a Hard-Margin SVM, support samples are samples that lie exactly on the margin boundaries. Mathematically, in the Hard Margin case, we can say that for any given support sample $(x_s, y_s)$, the following holds true 
    \begin{align}
    \label{eq:1a-2}
        y_s (\theta^T x_s + \theta_0) = 1
    \end{align}

    % https://kuleshov-group.github.io/aml-book/contents/lecture13-svm-dual.html#the-dual-of-the-svm-problem
    % https://machine-learning-upenn.github.io/assets/notes/Lec8.pdf
    % https://www.cs.cmu.edu/~aarti/Class/10701_Spring21/Lecs/svm_dual_kernel_inked.pdf

    % https://chatgpt.com/c/68bb1aad-1140-832b-918c-fa8c29b1b32d
    % https://www.csie.ntu.edu.tw/~cjlin/papers/libsvm.pdf
    \noindent In the Soft Margin case, support samples can lie on the margin, within the margin boundaries, or even be a misclassified point. 
    \item[(c)] Check Code
    
    %C-SVM:https://www.csie.ntu.edu.tw/~cjlin/papers/libsvm.pdf
    % C-Parameter: 
    % - https://eitca.org/artificial-intelligence/eitc-ai-mlp-machine-learning-with-python/support-vector-machine/svm-parameters/examination-review-svm-parameters/what-is-the-purpose-of-the-c-parameter-in-svm-how-does-a-smaller-value-of-c-affect-the-margin-and-misclassifications/
    % - https://scikit-learn.org/stable/modules/svm.html#kernel-functions
    % - https://scikit-learn.org/stable/auto_examples/svm/plot_rbf_parameters.html#sphx-glr-auto-examples-svm-plot-rbf-parameters-py
    \item[(d)] $C$ is a regularization parameter that controls the tradeoff between correctly classifying training points and maximizing the margin. A larger $C$ strongly penalizes misclassified points, favoring correct classification even if it results in a narrower margin. This can reduce generalization, making the model more sensitive to the training data. Conversely, a smaller $C$ tolerates some misclassifications, resulting in a wider margin and even better generalization. \\

    \noindent $\gamma$ is the kernel coefficient in rbf, polynomial, and sigmoid kernels, determining the reach of each training point's influence. This influence refers to a region around a training point where the kernel function assigns significant weight when computing similarities. A small $\gamma$ gives each training data point a broad influence, producing a smoother, more generalized decision boundary and better overall generalization. On the other hand, a larger $\gamma$ restricts each point's influence to a narrow area, enabling the model to fit the training data closely, which can increase overfitting and reduce generalization. \\  

    In Scikit-learn, the default value of $C$ is $1.0$, and the default value of $\gamma$ is "scale", which is computed as $\frac{1}{n_{\text{features}} * X.var()}$, where $n_{\text{features}}$ is the number of features and $X.var()$ is the variance of all the values in the design matrix. \\

    When running the given SVM Classifier, here are the experimental results \\
    \textbf{Experimental Results:}
    \begin{itemize}
        \item Training Accuracy: 
        \item Training Error: 
        \item Validation Accuracy: 
        \item Validation Error: 
        \item Support Vector Ratio: 
        \item Test Accuracy: 
        \item Test Error: 
    \end{itemize}

    % \begin{figure}[h!]
    %     \centering
    %     \includegraphics[width=0.5\linewidth]{confusion_matrix_1d.png}
    %     \caption{Confusion Matrix}
    %     \label{fig:placeholder}
    % \end{figure}

% \textbf{Identifying the Pattern:} \\
% From the confusion matrix, we can see that the SVM classifier predicts only $9$ for all test samples, which is the clear pattern here.  \\

% \textbf{Intuitive Explanation: } \\
% To understand this behavior, we examine the classifier's hyperparameters, starting with $\gamma$. By default, $\gamma = \tfrac{1}{n_{\text{features}} \cdot X.var()}$, but in our case we set $\gamma = \text{auto}$, which instead uses $\gamma = \tfrac{1}{n_{\text{features}}}$. For our dataset, $X.var() \approx 5164$, so choosing $\gamma = \text{auto}$ made our $\gamma$ value about $5164$ times larger than the default. From our earlier discussion, we know that having a higher $\gamma$ value restricts each point's influence to a narrow area, enabling the model to fit the training data closely, which can increase overfitting and reduce generalization. \\

\noindent \textbf{Extra: Improving Performance by Tweaking $\gamma$ to $\text{scale}$} \\
\textbf{Experimental Results:}
    \begin{itemize}
        \item Training Accuracy: 
        \item Training Error: 
        \item Validation Accuracy: 
        \item Validation Error: 
        \item Support Vector Ratio: 
        \item Test Accuracy: 
        \item Test Error: 
    \end{itemize}

% \begin{figure}[h!]
%     \centering
%     \includegraphics[width=0.5\linewidth]{confusion_matrix_1d_2.png}
%     \caption{Confusion Matrix}
%     \label{fig:placeholder}
% \end{figure}

\item[(e)] We will list out the parameters from Scikit-learn and analyze each of them: 
\begin{itemize}
    \item C: Already discussed previously
    \item kernel: Kernel function to use 
    \item degree: the degree of the polynomial kernel 
    \item gamma: Already discussed previously
    \item coef0: Independent term in polynomial and sigmoid kernels
    \item shrinking: 
    \item probability: 
    \item tol:
    \item $\text{cache\_size}$: Size of kernel cache in MB
    \item verbose: 
    \item $\text{max\_iter}$
    \item $\text{decision\_function\_shape}$
    \item $\text{break\_ties}$
    \item $\text{random\_state}$: 

\end{itemize}
    
\end{enumerate}
\end{solution}

\clearpage
% https://stanford.edu/class/ee364b/lectures/subgradients_notes.pdf
\begin{solution}[Time spent: 20 minutes] \\
Since $\varphi$ is convex and finite, its subdifferential (i.e., the set of all subgradients) at $\mu$ is defined as
\[
\partial \varphi(\mu) := \{\, y \in \mathbb{R}^n : \varphi(x) \geq \varphi(\mu) + y^\top (x - \mu) \;\; \text{for all } x \in \mathbb{R}^n \,\},
\]
and is guaranteed to be \textbf{\underline{nonempty}}, provided that $\mu \in \operatorname{int}(\operatorname{dom}\varphi)$. \\

\noindent \textit{Note: If $\mu$ lies on the boundary of the domain, the subgradient inequality is trivial, since the right-hand side can go to $-\infty$. Therefore, for the remainder of the proof, we assume $\mu \in \operatorname{int}(\operatorname{dom}\varphi)$.} \\



\noindent We begin by denoting an arbitrary subgradient $g \in \partial \varphi(\mu)$. By definition of a subgradient, we have
\[
\varphi(x)\ge \varphi(\mu)+ g^\top \left(x-\mu\right)
\]
for all $x \in \mathbb{R}^n$ \\

Applying the above inequality to the random vector \(X\) gives us the following
\[
\varphi(X)\ge \varphi(\mu)+ g^\top \left(X-\mu\right)
\]

Taking Expectation of both sides of the inequality gives us 
\[
\mathbf{E} \left[ \varphi(X) \right]\ge \mathbf{E} \left[ \varphi(\mu)+ g^\top \left(X-\mu\right) \right]
\]

Using Linearity of Expectation, we see that
\[
\mathbf{E} \left[ \varphi(X) \right]\ge \mathbf{E} \left[ \varphi(\mu) \right] + \mathbf{E} \left[ g^\top \left(X-\mu\right) \right]
\]

We use Linearity of Expectation on $\mathbf{E} \left[ g^\top \left(X-\mu\right) \right]$ to further simplify
\[
\mathbf{E} \left[ \varphi(X) \right]\ge \mathbf{E} \left[ \varphi(\mu) \right] + g^\top \cdot \mathbf{E} \left[\left(X-\mu\right) \right]
\]

Using Linearity of Expectation on $\mathbf{E} \left[\left(X-\mu\right) \right]$, we simplify 
\[
\mathbf{E} \left[ \varphi(X) \right]\ge \mathbf{E} \left[ \varphi(\mu) \right] + g^\top \cdot \left( \mathbf{E} \left[X \right] - \mathbf{E} \left[\mu \right] \right)
\]

Finally, we use the fact that $\varphi(\mu)$ and $\mu$ are deterministic and that \(\mathbb{E}[X]=\mu\) to simplify
\[
\mathbf{E} \left[ \varphi(X) \right]\ge \varphi(\mu) + g^\top \cdot \left( \mu - \mu \right)
\]
\[
\mathbf{E} \left[ \varphi(X) \right]\ge \varphi(\mu) + g^\top \cdot \left( 0 \right)
\]
\[
\mathbf{E} \left[ \varphi(X) \right]\ge \varphi(\mu)
\]

This proves the claimed inequality.

\end{solution}

\clearpage
\begin{solution}[Time spent: 1 hour]
\begin{enumerate}
    \item[(a)]
\end{enumerate}
\end{solution}

\end{document}